%%%%%%%%%%%%%%%%%%%%% chapter.tex %%%%%%%%%%%%%%%%%%%%%%%%%%%%%%%%%
%
% sample chapter
%
% Use this file as a template for your own input.
%
%%%%%%%%%%%%%%%%%%%%%%%% Springer-Verlag %%%%%%%%%%%%%%%%%%%%%%%%%%
%\motto{Use the template \emph{chapter.tex} to style the various elements of your chapter content.}
\chapter{Formatação de texto}
\label{intro} % Always give a unique label
% use \chaptermark{}
% to alter or adjust the chapter heading in the running head
Em todo texto é necessário elementos para format\'a-lo de acordo com as necessidades de cada escritor. Assim,
neste capítulo, apresentaremos os tópicos básicos para formatação de qualquer texto em \LaTeX{}.

\section{Tamanho e estilo de fonte}
\label{sec:1}
Um dos aspectos mais importantes quando se está formatando um texto é o tamanho da fonte. Abaixo, iremos mostrar as principais formas de fazer essas alterações em um texto utilizando o \LaTeX{}.\\

\begin{center}
	\begin{tabular}{l11}
		\hline
	\textbf{Comando}	&  \textbf{Resultado}\\
		\hline
	\verb|\tiny{Exemplo}| &  \tiny Exemplo\\ 
		\hline
	\verb|\scriptsize{Exemplo}|	& \scriptsize Exemplo \\
		\hline
	\verb|\footnotesize{Exemplo}|	&  \footnotesize Exemplo\\
		\hline
	\verb|\small{Exemplo}|	& \small Exemplo \\
		\hline
	\verb|\normalsize{Exemplo}|	& \normalsize Exemplo \\
		\hline
	\verb|\large{Exemplo}|	& \large Exemplo \\
		\hline
	\verb|\Large{Exemplo}|	& \Large Exemplo \\
		\hline
	\verb|\LARGE{Exemplo}|	& \LARGE Exemplo \\
		\hline
	\verb|\huge{Exemplo}|	& \huge Exemplo \\
		\hline
	\verb|\Huge{Exemplo}|	& \Huge Exemplo \\
		\hline
	\end{tabular}
\end{center}

\noindent Para deixar o texto em negrito é necessário usar o comando \verb|\textbf|. Veja o exemplo:
\begin{trailer}{Texto em negrito}

\begin{verbatim}\begin{document}
    \textbf{Exemplo}
\end{documument}\end{verbatim}
\end{trailer}

\noindent Para deixar o texto em itálico é necessário usar o comando \verb|\textit|. Veja o exemplo:
\begin{trailer}{Texto em itálico}
\begin{verbatim}\begin{document}
    \textit{Exemplo}
\end{document}\end{verbatim}
\end{trailer}

\noindent Para deixar o texto sublinhado é necessário usar o comando \verb|\underline|. Veja o exemplo:
\begin{trailer}{Texto sublinhado}
\begin{verbatim}\begin{document}
    \underline{Exemplo}
\end{document}\end{verbatim}
\end{trailer}

\noindent Para deixar o texto com alguma cor diferente é necessário que no preâmbulo esteja o pacote \verb|xcolor| e em seguida no texto com o comando \verb|\textcolor{}{}|, sendo o primeiro argumento a cor que deseja e no segundo o texto que deseja colorir. Veja o exemplo a seguir:
\begin{trailer}{Cor no texto}
\begin{verbatim}
\usepackage{xcolor}  
\begin{document}
    \textcolor{red}{Texto}
\end{document}
\end{verbatim}
\end{trailer}

%COMO FICARIA 
%
\section{Espaçamento entre linhas}
Para alterar o espaçamento entre linhas é preciso adicionar no preâmbulo do texto o pacote \verb|setspace| e no  corpo do texto adicionar um dos seguintes comandos:
\begin{itemize}
    \item \verb|\singlespacing| - Espaçamento simples
    \item \verb|\onehalfspacing| - Espaçamento de $1,5$
    \item \verb|\doublespacing| - Espaçamento duplo
\end{itemize}

\begin{trailer}{Espaçamento entre linhas}
\begin{verbatim}
\usepackage{setspace}  
\begin{document}
    \singlespacing
    Copiar texto desejado. \\
    O texto terá espaçamento simples 

    %Para alterar espaçamento basta adicionar o outro

    \onehalfspacing
    Copiar texto desejado.\\
    O texto terá espaçamento de 1,5
\end{document}
\end{verbatim}
\end{trailer}

\noindent Se você deseja alterar o espaçamento apenas para um trecho específico de texto (e não para todo o documento), você pode usar os ambientes \verb|singlespace|, \verb|onehalfspace| e \verb|doublespace|. 

\section{Hifenização}
Em um texto em \LaTeX{} a separação silábica é feita de forma automática, visto que a linguagem já é automatizada para que possa ser feito em qualquer idioma. Basta utilizar no preâmbulo o pacote \verb|\usepackage[portuguese]{babel}|. No entanto, caso se queira fazer a hifenização de forma manual de uma palavra é preciso utilizar o comando \verb|\hyphenation{e-xem-plo}| no preâmbulo. 


\section{Ajuste de margens}
Em um texto em \LaTeX{} é preciso ajustar as margens de acordo com as necessidades de cada escritor. Assim, para configurar as margens é necessário inserir no preâmbulo  o pacote \verb|\usepackage{geometry}|. Veja o exemplo abaixo:
\begin{trailer}{Ajustando margens}
\begin{verbatim}
\usepackage[left=2cm,right=2cm,top=3cm,bottom=3cm]{geometry}
%Defina a configuração das margens

\begin{document}
   Inserir texto.
\end{document}
\end{verbatim}
\end{trailer}

\section{Alinhamento do texto}
O alinhamento em um texto é importante, pois distrubui o texto uniformemente entre as margens. Para se ter um alinhamento à esquerda, à direita ou centralizado em um trecho do texto é preciso inserir o texto dentro dos ambientes citados abaixo.
\begin{trailer}{Alinhamento}
\begin{verbatim}
\begin{document}
    \begin{flushleft}
    Trecho alinhado à esquerda
    \end{flushleft}

    \begin{flushright}
    Trecho alinhado à direita
    \end{flushright}

    \begin{center}
    Trecho centralizado
    \end{center}
\end{document}
\end{verbatim}
\end{trailer}

\noindent Para se fazer o alinhamento do texto todo é necessário inserir os seguintes comandos antes do texto:

\begin{trailer}{Alinhamento em todo texto}
\begin{verbatim}
\begin{document}
\raggedleft   %alinhamento à esquerda
\raggedright  %alinhamento à direita
\centering    %centralizado
%Selecione algum desses alinhamentos e inicie a escrita.

Escrever texto
\end{document}
\end{verbatim}
\end{trailer}

\noindent Para que o texto seja escrito de forma justificada, é necessário inserir no preâmbulo o pacote \verb|ragged2e| e utilizar o comando \verb|\justifying| antes de iniciar 
a escrita. Veja o exemplo abaixo:

\begin{trailer}{Alinhamento}
\begin{verbatim}
\usepachege{ragged2e}
\begin{document} \justifying 

Escrever texto que será justificado.

\end{document}
\end{verbatim}
\end{trailer}

\section{Inserir cabeçalho, rodapé e número de páginas}
Em diversos textos, é necessário que tenha cabeçalho, rodapé e indica\c c\~ao do número de páginas para uma melhor organização do texto. Na escrita \LaTeX{} essas configurações são feitas de forma simples e rápida.

\subsection{Cabeçalho e rodapé}
Para inserir cabeçalho e rodapé no texto em \LaTeX{} é necessário inserir no preâmbulo o pacote \verb|fancyhdr| e, em seguida, ainda no preâmbulo, colocar\verb|\pagestyle{fancy}| para que o estilo da página siga a fórmula dada pelo pacote \verb|fancyhdr|. Veja o exemplo

\begin{trailer}{Cabeçalho e rodapé}
\begin{verbatim}
\pagestyle{fancy} % Define o estilo de página como "fancy"
\fancyhf{} % Limpa todos os estilos de cabeçalho e rodapé atuais

% Configuração do cabeçalho
\fancyhead[L]{Cabeçalho à esquerda}
\fancyhead[C]{Cabeçalho no centro}
\fancyhead[R]{Cabeçalho à direita}

% Configuração do rodapé
\fancyfoot[L]{Rodapé à esquerda}
\fancyfoot[C]{Rodapé no centro}
\fancyfoot[R]{Rodapé à direita}

% Linha superior do cabeçalho
\renewcommand{\headrulewidth}{0.4pt}

% Linha inferior do rodapé
\renewcommand{\footrulewidth}{0.4pt}
\end{verbatim}
\end{trailer}

\noindent Para fazer as configurações para o estilo, podemos usar os comandos \verb|lhead|, \verb|rhead|, \verb|lfoot|, \verb|cfoot| e \verb|rfoot| para que seja possível definir o conteúdo que vai ser inserido no cabeçalho e no rodapé em páginas distintas do texto. Já para inserir em uma página específica, basta usar o comando \verb|pagestyle{fancy}|. Veja o exemplo de como colocar cabeçalho e rodapé:

\begin{trailer}{Exemplo}
\begin{verbatim}
\usepackage{fancyhdr}
\pagestyle{fancy}
\fancyhf{}
\fancyhead[L]{Cabeçalho à esquerda}
\fancyhead[C]{Cabeçalho no centro}
\fancyhead[R]{Cabeçalho à direita}
\fancyfoot[L]{Rodapé à esquerda}
\fancyfoot[C]{Rodapé no centro}
\fancyfoot[R]{Rodapé à direita}
\renewcommand{\headrulewidth}{0.4pt}
\renewcommand{\footrulewidth}{0.4pt}
\begin{document}

 Seu conteúdo vai aqui.

\end{document}
\end{verbatim}
\end{trailer}

\subsection{Número de página}
Para inserir número nas páginas no texto em \LaTeX{}, é preciso inserir no preâmbulo o pacote \verb|fancyhdr|, já visto anteriormente. Em seguida, para exibir os números nas páginas, basta inserir o comando \verb|\thepage|. Veja exemplo:
\begin{trailer}{Número nas páginas}
\begin{verbatim}
\usepackage{fancyhdr}

\pagestyle{fancy}
\fancyhf{}
\fancyhead[L]{Cabeçalho à esquerda}
\fancyhead[C]{Cabeçalho no centro}
\fancyhead[R]{Cabeçalho à direita}
\fancyfoot[L]{Rodapé à esquerda}
\fancyfoot[C]{\thepage} % Insere o número da 
página no centro do rodapé
\fancyfoot[R]{Rodapé à direita}
\renewcommand{\headrulewidth}{0.4pt}
\renewcommand{\footrulewidth}{0.4pt}

\begin{document}

Seu conteúdo vai aqui.

\end{document}

\end{verbatim}
\end{trailer}

\noindent Neste exemplo, o comando \verb|\fancyfoot[C]{\thepage}| insere o número da página no centro do rodapé.


