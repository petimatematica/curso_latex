%%%%%%%%%%%%%%%%%%%%%%acknow.tex%%%%%%%%%%%%%%%%%%%%%%%%%%%%%%%%%%%%%%%%%
% sample acknowledgement chapter
%
% Use this file as a template for your own input.
%
%%%%%%%%%%%%%%%%%%%%%%%% Springer %%%%%%%%%%%%%%%%%%%%%%%%%%

\extrachap{Agradecimentos}

Os autores agradecem ao Programa de Educa\c c\~ao Tutorial Institucional da Univesidade Estadual do Sudoeste da Bahia (PETI/UESB) pelas bolsas de estudo e à \textit{Springer Nature} por disponibilizar o template que tornou poss\'ivel a cria\c c\~ao deste material. 

\extrachap{Apresentação}
\noindent O Programa de Educa\c c\~ao Tutorial Institucional de Matem\'atica (PETIMAT) tem como miss\~ao aprimorar os cursos regulares de gradua\c c\~ao na Universidade Estadual do Sudoeste da Bahia (UESB), com o objetivo de proporcionar uma forma\c c\~ao abrangente e de alta qualidade para os alunos envolvidos, mantendo a integra\c c\~ao entre ensino, pesquisa e extens\~ao como um princ\'ipio fundamental do ambiente universit\'ario.

\noindent O PETIMAT se dedica a elevar o padr\~ao da forma\c c\~ao dos estudantes de gradua\c c\~ao e promover o sucesso acad\^emico. Para alcan\c car esses objetivos, o programa oferece, a cada semestre, minicursos voltados principalmente para os alunos matriculados no curso de matem\'atica da UESB. Esses minicursos visam estimular a capacita\c c\~ao de futuros profissionais e docentes, proporcionando-lhes uma qualificação acad\^emica, cient\'ifica, t\'ecnica e tecnol\'ogica. 

\noindent Dentro desse contexto, o PETIMAT desenvolveu o minicurso "Introdução \`a Edi\c c\~ao de Textos usando LaTeX". O objetivo desse minicurso \'e apresentar conceitos e no\c c\~oes iniciais relacionados ao LaTeX, bem como fornecer instru\c c\~oes sobre comandos b\'asicos de formata\c c\~ao de texto e introduzir os principais tipos de documentos cient\'ificos, tais como artigos, monografias, apresenta\c c\~oes (utilizando o beamer) e p\^osteres. A inten\c c\~ao \'e capacitar os participantes para que se tornem aut\^onomos na produ\c c\~ao de seus pr\'oprios textos.

\noindent Este material corresponde \`as notas de aula e serve como um recurso de consulta para os participantes do minicurso. Nesta apostila, est\~ao dispon\'iveis informações detalhadas e comandos necess\'arios para criar e personalizar esses diversos tipos de documentos. Para ter acesso aos diferentes modelos de documentos criados durante o minicurso, basta consultar o nosso reposit\'orio no GitHub:
\begin{center}
    \url{https://github.com/petimatematica/curso_latex}
\end{center}
\noindent Para conhecer mais sobre o trabalho e as publica\c c\~oes do PETIMAT, acesse o nosso site: 
\begin{center}
\url{http://www2.uesb.br/programa/petimatematica/}
\end{center}
e nossa conta oficial no Instagram: 
\begin{center}
\url{https://www.instagram.com/petimatuesb/}
\end{center}
onde est\~ao dispostas informa\c c\~oes detalhadas sobre nossas atividades, projetos, eventos e conte\'udos relacionados ao nosso programa.


